\documentclass[9pt,twocolumn]{extarticle}

\usepackage[hmargin=0.5in,tmargin=0.5in]{geometry}
\usepackage{amsmath,amssymb}
\usepackage{times}
\usepackage{graphicx}
\usepackage{subfigure}

\usepackage{cleveref}
\usepackage{color}
\newcommand{\TODO}[1]{\textcolor{red}{#1}}

\newcommand{\FPP}[2]{\frac{\partial #1}{\partial #2}}
\newcommand{\argmin}{\operatornamewithlimits{arg\ min}}
\author{Siwang Li}

\title{Solve LCP with Extened MPRGP}

%% document begin here
\begin{document}
\maketitle

\setlength{\parskip}{0.5ex}

\section{LCP}
\begin{equation}
  A x = b \quad \mbox{ s.t. } \quad J x \ge c
\end{equation}

\section{MPRGP Extension}
\subsection{Projection}
The projection process is to find an vector $y$ which satisfies the constraints while is cloest to the un-projected vector $x$:
\begin{equation}
  P_{\Omega}(x) = \min_{y}\frac{1}{2}\|y-x\|_2^2 \quad \mbox{ s.t. } \quad J y \ge c
\end{equation}
Considering its KKT condition, we have
\begin{equation}
  y = x + J^T\lambda  \quad \mbox{ s.t. } \quad 0 \le \lambda \perp J y \ge c
\end{equation}
We can solve the following LCP sub problem to obtain $\lambda$, 
\begin{equation}
   0 \le \lambda \perp (JJ^T)\lambda \ge c - Jx
\end{equation}
\TODO{which is equal to solving (why?)}
\begin{equation}
  (JJ^T)\lambda=c - Jx \quad \mbox{s.t.} \quad \lambda \ge 0
\end{equation}
As $J^TJ$ is a \TODO{SPD matrix (why?)}, we can use traditional MPRGP solver to solve it. Further more, as $J$ is sparse and constant, we can use Incomplete Cholesky Factorization as a pre-conditioner. \TODO{(But how to combine it with MPRGP?)}

\subsection{Step Limit}
Given a feasible point $x$ and a direction $p$, the step limit $\alpha$ is a positive scalar which satisfies
\begin{equation}
  J(x+\alpha p) \ge c
\end{equation}
Thus we can compute $\alpha$ by using
\begin{equation} 
 \alpha = \min_{\alpha} \alpha_i 
\end{equation}
where
\begin{equation}
  \alpha_i = \left\{
  \begin{array}{l l}
    +\infty & \quad \text{if $J_ip \ge 0$}\\
    \frac{c_i-J_ix}{J_ip} & \quad \text{if $J_ip < 0$}
  \end{array} \right  
\end{equation}
Here, $J_i$ is the $i$-th row of matrix $J$.

\subsection{Free gradient}
We suppose the constraints active constraints is given by $\hat{J}x = c$, then the free gradient $\phi(x)$ is in the NULL space of $\hat{J}$ and should be as close to the gradient $g = Ax-b$ as possible:
\begin{equation}
  \phi = \min_\phi \frac{1}{2} \|\phi - g\|_2^2  \quad \mbox{ s.t. } \quad \hat{J} \phi = 0
\end{equation}
The KKT condition for this problem is
\begin{equation}
  \phi = g + \hat{J}^T \lambda \quad { s.t. } \quad \hat{J}\phi = 0
\end{equation}
we can obtain $\lambda$ by solving
\begin{equation}
  (\hat{J}\hat{J}^T) \lambda = -\hat{J}g
\end{equation}
Here, as $\hat{J}\hat{J}^T$ is a SPD matrix, we can use Conjugate Gradient method to solve this problem. However, as $\hat{J}^T\hat{J}$ is not constant, we need a more cheap preconditioner rather than Incomplete Cholesky Factorization, such as $SOR$ preconditioner. (\TODO{Any more efficient method?})

\subsection{Chopped gradient}
The chopped gradient $\beta(x)$ should be complemented to the free direction $\phi(x)$, and it should be as close to the gradient $g$ as possible. What's more, it is required that $x+\alpha \beta$ should always satisfies the constraints for any $\alpha > 0$. Thus we compute $\phi(x)$ by solving
\begin{equation}
  \beta = \min_\beta \frac{1}{2} \|\beta - g\|_2^2  \quad \mbox{ s.t. } \quad \phi^T \beta = 0 \mobx{, }\quad \hat{J} \beta \ge 0
\end{equation}
The KKT condition of this problem is
\begin{equation}
  \beta = g + \phi \lambda_1 + \hat{J}^T\lambda_2  \quad \mbox{ s.t. } \quad \phi^T \beta = 0 \mobx{, }\quad 0 \le \lambda_2 \perp \hat{J} \beta \ge 0
\end{equation}
By replacing $\beta$ with $\lambda_1$, $\lambda_2$ in the constraints, and using the fact that $\hat{J}\phi = 0$, we obtain
\begin{equation}
  \phi^T\phi \lambda_1 = -\phi^Tg
\end{equation}
\begin{equation}
  \hat{J}\hat{J}^T\lambda_2 = -\hat{J}g \quad \mbox{s.t.} \quad \lambda_2 \ge 0
\end{equation}
As $\hat{J}^T\hat{J}$ is a \TODO{SPD matrix (why?)}, we can use traditional MPRGP solver to solve for $\lambda_2$, and \TODO{use SOR as preconditioner}.

%% references
% \begin{thebibliography}{99}
% \bibitem{sig2011} Fast simulation of skeleton-driven deformable body
%   characters.
% \end{thebibliography}

\end{document}
