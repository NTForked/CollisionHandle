\documentclass[9pt,twocolumn]{extarticle}

\usepackage[hmargin=0.5in,tmargin=0.5in]{geometry}
\usepackage{amsmath,amssymb}
\usepackage{times}
\usepackage{graphicx}
\usepackage{subfigure}

\usepackage{cleveref}
\usepackage{color}
\newcommand{\TODO}[1]{\textcolor{red}{#1}}

\newcommand{\FPP}[2]{\frac{\partial #1}{\partial #2}}
\newcommand{\argmin}{\operatornamewithlimits{arg\ min}}
\author{Siwang Li}

\title{Friction}

%% document begin here
\begin{document}
\maketitle

\setlength{\parskip}{0.5ex}

\section{Introduction}
Firstly, we compute the collision forces by using the last status,
\begin{equation}
  f^c = A'x' - b'
\end{equation}
Then for each collided node $i$, we compute its fractional forces by using
\begin{equation}
  f^r_i = -\mu \|f^c_i\|_2 \frac{v'_i}{\|v'_i\|}
\end{equation}
where $v'_i$ is the volicity of node $i$ in the last time step.

\end{document}
